\chapter{Woordenlijst}
\begin{tabular}{|p{3cm}|p{11cm}|}
	\hline
	\textbf{Word} & \textbf{Meaning} \\
	\hline
	Appsemble & Een interactieve webapp van d-centralize om apps te ontwikkelen\\ \hline
	CUDA & Programmeertaal voor berekeningen te doen op grafische kaarten \\ \hline
	AWS & Amazon Web Services, de amazon cloud service provuder \\ \hline
	GAE & Google App Engine, de google cloud service provider \\ \hline
	Extension & Woord gebruikt in appsemble om een onderdeel van een app te beschrijven \\ \hline
	Library & Een stukje code dat door andere developers gebruikt kan worden om verschillende functies standaard aan te bieden. \\ \hline
	Developer & Ontwikkelaar, software engineer. \\ \hline
	Backend & Het achterliggende stuk software, wat een gebruiker niet ziet. \\ \hline
	Frontend & De interface die een gebruiker wel ziet. \\ \hline
	CLI & Command Line Interface. Een programma dat aangeroepen word vanaf de commandoregel. \\ \hline	
	ES6 & Ecmascript 6. De nieuwe standaard ecmascript (Ook wel bekend als javascript). Deze versie is niet ondersteund in iedere browser. \\ \hline
	Promise & Een concept bij Javascript, waarbij de code een belofte maakt om later een stuk code uit te voeren indien de operatie een success is. Zorgt voor simpele manier om asynchrone code uit te voeren. \\ \hline
	SDK & Software Development Kit, code die gebruikt kan worden om nieuwe extensions te schrijven voor Appsemble. \\ \hline
	NPM & Node Package Manager. Een package manager voor javascript. Vergelijkbaar met NuGet, Maven, Gradle, Pip en Gem.
\end{tabular}