\chapter{Woordenlijst}
\begin{tabular}{|p{3cm}|p{11cm}|}
	\hline
	\textbf{Word} & \textbf{Meaning} \\
	\hline
	Appsemble & Een interactieve webapp van d-centralize om apps te ontwikkelen\\ \hline
	CUDA & Programmeertaal voor berekeningen te doen op grafische kaarten \\ \hline
	AWS & Amazon Web Services, de amazon cloud service provuder \\ \hline
	GAE & Google App Engine, de google cloud service provider \\ \hline
	Extension & Woord gebruikt in appsemble om een onderdeel van een app te beschrijven \\ \hline
	Library & Een stukje code dat door andere developers gebruikt kan worden om verschillende functies standaard aan te bieden. \\ \hline
	Developer & Ontwikkelaar, software engineer. \\ \hline
	Backend & Het achterliggende stuk software, wat een gebruiker niet ziet. \\ \hline
	Frontend & De interface die een gebruiker wel ziet. \\ \hline
	CLI & Command Line Interface. Een programma dat aangeroepen word vanaf de commandoregel. \\ \hline	
	ES6 & Ecmascript 6. De nieuwe standaard ecmascript (Ook wel bekend als javascript). Deze versie is niet ondersteund in iedere browser. \\ \hline
	Promise & Een concept bij Javascript, waarbij de code een belofte maakt om later een stuk code uit te voeren indien de operatie een success is. Zorgt voor simpele manier om asynchrone code uit te voeren. \\ \hline
	SDK & Software Development Kit, code die gebruikt kan worden om nieuwe extensions te schrijven voor Appsemble. \\ \hline
	NPM & Node Package Manager. Een package manager voor javascript. Vergelijkbaar met NuGet, Maven, Gradle, Pip en Gem. \\ \hline
	Refactor & Het herschrijven van een Applicatie \\ \hline
	Closed Source & Software waarvan de broncode niet publiekelijk beschikbaar is \\ \hline
	Open Source & Software waarvan iedereen de broncode kan inspecteren en verspreiden. \\ \hline
	API Call & Een software functie die een aanvraag doet bij een API \\ \hline
	Roadmap & Een lijst van issues die samen een release of product vormen \\ \hline
	Framework & Een collectie van functies die samen een geheel vormen. In tegenstelling tot een library biedt een framework meer  opties (Een webapp framework tegen een library voor HTML templating) \\ \hline
	Linting & Code statisch analyseren voor bugs en stijlbreuken \\ \hline
	Merge & Code van twee verschillende bronnen samenvoegen tot een enkel stuk code \\ \hline
	IFrame & Een HTML element waar een andere pagina in de huidige pagina geladen kan worden. \\ \hline
	Wrapper & Een stuk code die een gecompliceerder stuk code verstopt \\ \hline
	Non-Blocking & Code die nergens een thread blokeert \\ \hline
	Docker & Containersoftware voor linux \\ \hline
\end{tabular}