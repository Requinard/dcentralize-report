\chapter{Samenvatting}
Dit document bevat de beschrijving van de verschillende onderdelen in de appsemble sdk en hoe deze tot stand zijn gekomen. Bij ieder onderdeel is het doel om aan te geven hoe en waarom bepaalde design-keuzes gemaakt zijn. \\

Het product zelf is de appsemble sdk, een hulpmiddel voor het ontwikkelen van extensions voor appsemble. Dit bestaat uit een library voor developers en een backend die functies van de library uitvoert.

\section{Backend}

De backend is het gedeelte van de sdk die alle aanvragen afhandelt. Het zorgt voor de validatie en communicatie tussen developer en appsemble. Deze code zit in appsemble ingebakken en is niet zichtbaar voor developers. 

\section{Libraries}

Developers maken uiteindelijk gebruik van de front-end libraries. Deze verbinden makkelijk te lezen functies met de obscure appsemble sdk interfaces. Het zorgt ervoor dat developers zelf geen api calls uit hoeven te voeren, enkel een functie.

\section{Developer Tool}

Als laatste zijn er nog een paar taken over die het leven van een developer gemakkelijker maken. Hiervoor is de cli tool. Dit programma automatiseert de complexere taken om ervoor te zorgen dat een developer kan ontwikkelen zonder moeite.