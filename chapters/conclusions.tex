\chapter{Conclusie}

\section{Uitgevoerd werk}
Het project is uiteindelijk goed afgesloten. De code waaraan is gewerkt  is gemerged met de algemene codebase en word in productie gebruikt. \\

Daarnaast is de geschreven code grondig geinspecteerd, goed getest en meerdere malen herschreven om kwaliteit te garanderen. \\

Daarnaast is er ook veel discussie geweest over verschillende aspecten van het platform zelf, leidend tot een constante groei van het platform in het algemeen. \\

Op het moment word de code live gebruikt voor appsemble. Daarnaast zijn de libraries en de cli ook gemakkelijk te downloaden van npm om zelf te gebruiken. 

\section{Behaalde resultaten}

Aan het eind van de stage is het doel om een werkend product op te leveren. In dit geval betekent dit dat een developer eenextensie kan ontwikkelen voor het appsemble platform. \\

Dit doel is gemakkelijk bereikt. Er is een tool ontwikkeld om gemakkelijk een extension te starten en te uploaden. Dit zorgt ervoor dat developers extensies kunnen publiceren. \\

Daarnaast is er ook de mogelijheid om gebruik te maken van de SDK. Deze zorgt ervoor dat extensions kunnen communiceren met het framework. Dit betekent dat er data opgeslagen kan worden en dat de extensions dinge van het framework kunnen vragen, zoals het openen van het menu of het aanroepen van native libraries.

\section{Aanbevelingen en advies}

D-centralize als bedrijf is een solide bedrijf met veel gezonde praktijken om een goede groep personeel te houden. Er word veel verzorgd en de developers hebben de vrijheid om zelf keuzes te maken, in tegenstelling tot bijzonderveel andere softwarebedrijven. \\

Daarnaast word innovatie gewaarborgd, vooral vanwege de keuzevrijheid. Dit betekent dat er moderne technologien worden gebruikt die de "cutting edge" van tegenwoordig zijn. \\

Echter betekent dit ook dat d-centralize de "Move fast and break stuff" mentaliteit aanhoudt. De gebruikte software heeft ook deze mentaliteit. Dit betekent dat de gebruikte libraries vaker niet volwassen zijn. Het voorbeeld hiervan is angular2, die bij aanvang van de stage pas een maand oud was. \\

Alhoewel dit niet inherent is, is support van deze libraries nooit verzekerd. Daarnaast is het ecosysteem, in tegenstelling tot java en C	\#, nog niet bijzonder ontwikkeld. \\

NPM bevat daarnaast veel libraries voor dezelfde doeleindes. Hierbij varieert de levensduur en kwaliteit van de libraries vaker. Vanwege het feit dat javascript standaard bijzonder weinig ondersteunt, moet er vaak een library gekozen worden. Vanwege de overweldigende keuze is dit echter een tijdrovend process. \\

Ik zou daarom aanraden om een standaard setje libraries op te stellen voor projecten zodat de software van d-centralize op hetzelfde niveau start, iedereen weet hoe de software en libraries gebruikt moeten worden. Dit zorgt voor een nette standaard voor het hele bedrijf. \\