\chapter{Evaluation}

\section{Opdracht}

Al met al was de opdracht uitdagend genoeg. In plaats van werken aan simpele database interfaces mocht ik aan de slag met een stuk software wat echt door andere developer gebruikt gaat worden. \\

Daarnaast is de code die ik hier geschreven heb een stuk complexer dan (naar mijn ervaring) andere stage-opdrachten. \\

Daarnaast werd de code ook niet ontwikkeld in een vacuuum. Ik moest samenwerken met andere developers en constant up to date blijven met heet gedane werk. \\ 

Alhoewel de samenwerking tussen mij en de lead developer, Remco, niet goed verliep, is de code toch uiteindelijk volgens iedereen het beste wat het kan zijn.

\section{Omgeving}

D-centralize als bedrijf is een leuke plek om te werken. De collega's zijn gezellig en hebben allemaal veel ervaring. De projecten zijn interressant en innoverend. Daarnaast maken deze projecten gebruik van allerlei nieuwe techonologieen, die bij Fontys of andere stageopdrachten niet te gebruiken zijn. \\

De collega's zijn naast vriendelijk ook zeer professioneel. Zij waren altijd aanspreekbaar over software problemen en gerelateerde issues. Daarnaast zijn het ook enthousiaste linux gebruikers en voorstanders van open source\footnote{En in een geval, free software}. \\

Daarnaast gebruikt het bedirjf veel manieren om de kwaliteit van de code te waarborgen. Alhoewel het vaker worstelen was met de testen en de buildserver\footnote{De bron van vele frustraties} was, zou ik dit niet negatief noemen. Ik heb ervan geleerd hoe je beter testbare code schrijft. \\

Daarnaast is het samenwerken met daadwerkelijk professionele collega's een geheel andere ervaring als het ontwikkelen met studenten. Hier word meer de nadruk gelegd op geode code schirjven in plaats van het schrijven van code die het net doet om de presentatie te halen. \\ 

Verder is het ook een culturele verandering om van schooluren naar bedrijfsuren te gaan. Waar bij fontys iedereen om 2 uur naar huis gaat, is dit op werk een stuk minder acceptabel. De eerste weken waren dan ook heel veel aanpassen aan het nieuwe werkklimaat. \\

Echter is het verschil met fontys dag en nacht. Hier word tijdens werk serieus gewerkt, in een rustige omgeving, met alle tools die je nodig kunt hebben. Een comfortable stoel, een groot bureau en 2 extra beeldschermen. Hier is eraan gedacht dat developer comfort een goed ding is. Dit in tegenstelling tot fontys, waar er bijna geen plek is, in een druk lokaal met constante onderbrekingen. \\

d-centralize moedigt ook nieuwe projecten en technologieen aan, in tegenstelling tot fontys. Hier word onderzoek en experimenteren beloond in plaats van afgestraft met extra werk. Developers zijn hier echt vrij om hun eigen keuzes te maken.