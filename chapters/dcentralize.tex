\chapter{D-centralize}
\section{what do they even do?}
d-centralize is een klein bedrijf in het hart van Eindhoven. Een van de drijfveren is het innoveren in de software markt. Dit doen zij doo aan meerdere producten tegelijkertijd te werken, in verschillende markten.

Dit betekent dat er op het moment 4 grote projecten onder development zijn, gespreid over 4 developers en 5 stagaires.

Gekoppeld met een informele en gezellige cultuur is het een bedrijf waar innovatie uit lekt. \\

Volgens de eigen woorden is d-centralize als volgt te beschrijven: "\textit{With a small and cozy office, their laptops and the bread-making machine, they are planning on making a mark in the software world.}" \cite{dcent1}

\section{Projects}

D-centralize ontwikkelt de volgende producten full-time.

\subsection{AppSemble}

Tegenwoordig hebben veel bedrijven een eigen app. Het maken hiervan is echter niet makkelijk, noch goedkoop. Bij websites is het probleem opgelost met applicaties als Wordpress en Joomla. Voor apps zijn de oplossingen echter moeilijk te vinden. 

Appsemble is een webapplicatie die beschikbaar is om met een simpele WYSIWYG\footnote{What you see is what you get. Het eindproduct zal er hetzelfde uitzien als hetgeen de ontwikkelaar ziet} interface. Dit betekent dat een klant snel een app kan prototypen, testen en distribueren, zonder ooit een regel code aan te raken.

De applicatie bestaat uit een framework, genoemd appsemble, en de bouwblokken, de extensions. Een grote verscheidenheid aan extensions zorgt ervoor dat men gemakkelijk een diverse app kan maken.

\subsection{ITSLanguage Speech API}

ITSLanguage is een platform voor het analyseren van uitspraak, en hier "on the fly"\footnote{Terwijl je spreekt} feedback op geven. Dit wordt gedaan door spraak te streamen naar de cloud, waar deze op GPU's geanalyseerd word. De resultaten hiervan worden met javascript zichtbaar gemaakt. \\

Het doel is om een geautomatiseerde assistent aan te bieden  bij het leren van andere talen. Op het mooment zijn de gebruikers dan  ook grote onderwijsinstellingen en hun leveranciers.

\subsection{DoeHetZelfContracten.nl}

DoeHetZelfContracten\footnote{Ook bekend als Cgen} is een website waar mensen zelf, zonder veel moeite, contracten in elkaar kunnen sleutelen. Dit doen zij met een simpele interface die zoveel mogelijk complexiteit en legalees uit de weg haalt. \\

Het wordt ontwikkeld in samenwerking met meerdere advocatenbureau's en is onderhand al bruikbaar en in beta. \footnote{Ga naar http://doehetzelfcontracten.nl}

\subsection{Pro6PP}

Tegenwoordig werken veel applicaties met je echte locatie. Adressen moeten echter gevalideerd worden. Dit gebeurt vaker door de combinatie van straat en postcode.

Echter is het niet zo simpel om gemakkelijk aan deze data te komen. Als een bedrijf dit zelf zou moeten bijhouden zouden er vele manuren in verloren gaan. Daarnaast verandert deze dataset van maand tot maand. \\

Pro6pp is een api die voorziet in de postcodes. Straten en locaties zijn gekoppeld aan de postcode. Daarnaast word er ook onderscheid gemaakt tussen belangrijke landmarks in het systeem. Hiermee kunnen bedrijven vergeten over het hele postcode-gedoe.

\section{Het appsemble team}

De volgende mensen werken bij d-centralize aan appsemble.

\begin{itemize}
	\item Remco Hassing			- Project lead
	\item Mathijs van den Worm	- Software Engineer
	\item David Diks			- Software stagaire (Frontend)
	\item Guus Hamm				- Software Stagaire (SDK)
\end{itemize}