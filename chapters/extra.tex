\chapter{Extra werk}

Nadat de basis van de SDK stond was de basis van de opdracht af. Dit betekent dat er extra werk stond, welke nog uitgevoerd moest worden. 

\section{Contact Extension}

Als eerste stond er een test van de SDK klaar. Het doel is het schrijven van een extension die gebruikt maakt van de SDK. Hiervoor is Contact vertaald naar Appsemble Next. De extension is nu modulair ontworpen en gebruikt de Appsemble SDK. \\

Het nieuwe component is ontwikkeld met ReactJS\footnote{Een web framework voor applicaties} en Redux\footnote{Een framework voor het bijhouden van state}. Daarnaast is er ook gebruik gemaakt van material-ui voor de frontend. \\

Hiervoor is gekozen omdat het een nieuwe en vooruitstrevende software stack\footnote{Collectie van software om applicaties mee te maken} is. Daarnaast is deze ook uiterst simpel om te testen en te deployen.

\section{Tutorial}

Een belangrijk component van een goede SDK is de documentatie. Voor de Appsemble SDK is ervoor gekozen om gebruik te maken van een tutorial en functie-documentatie. \\

De tutorial werkt als een opstapje die het Appsemble platform uitlegt. Er word een simpele applicatie opgezet en uitgewerkt waardoor er een referentie is voor andere developers. \\

Daarnaast word er ook gebruik gemaakt van function-documentation\footnote{Documentatie recht in de code, waar input en output gedefinieerd worden}. Dit zorgt ervoor dat de documentatie levend is en zal volgen zodra er veranderingen worden doorgevoerd, aangezien documentatie alleen opnieuw gegenereerd hoeft te worden.

\section{Bug en Usability fixes}

Als laatste zijn er nog vele kleine patches. Deze zullen inbegrepen zijn bij de printout van de gerrit changesets. Dit zijn allemaal kleine patches om bugs te repareren.